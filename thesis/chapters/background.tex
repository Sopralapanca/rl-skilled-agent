%! Author = giaco
%! Date = 16/05/2024

\section{Background}
\label{sec:background}
In this section, we provide a brief overview of the main concepts that are necessary to understand the work presented in this thesis.
First of all, we start by introducing the concept of neural networks and deep learning in \ref{sec:dl}, which is the foundation of most of the models used in this work.
Then, we introduce reinforcement learning in \ref{sec:rl}, which is instead the main focus of this work.
In this section, we provide a brief overview of the main concepts that are necessary to understand the work presented in this thesis. We will talk about model free reinforcement learning,

\subsection{Deep Learning}
\label{sec:dl}
A neural network is an artificial intelligence method that is inspired by the way the human brain works.
Every neural network, consists of layers of interconnected nodes, which are called neurons.
We have an input layer, which is responsible for receiving the input data, one or more hidden layers, that are responsible for processing the data in a way that encodes the information in a latent space, and finally an output layer, which is responsible for producing the output data.
Neurons are connected to each other through edges, which are weighted connections that are learned during the training process.

Non-linear activation functions are used to introduce non-linearity in the model, which allows the model to learn complex patterns in the data.

Backpropagation is the algorithm used to train the neural network.

%In a fully connected
%All the neurons in a layer are connected to all the neurons in the next layer.
%
%
%is used to recognize patterns in data.

\subsection{Reinforcement Learning}
\label{sec:rl}
Reinforcement Learning (RL) is a subfield of machine learning that focuses on training agents to make sequences of decisions in an environment to maximize a reward signal.

